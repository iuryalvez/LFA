\documentclass[12pt]{article}

% ----------------------------------------- Dados do discente
% Insira os seus dados e do exercício escolhido:
\def\discente{Fulana(o) de tal}
\def\matricula{20010101}
\def\ua{05}
\def\myling{{99}} % Informe o número da linguagem selecionada.

% ------------------------------------------ Babel & Geometry
\usepackage[brazil]{babel}
\usepackage[T1]{fontenc}
\usepackage[utf8]{inputenc}
\usepackage[a4paper,top=0.5cm,bottom=1.5cm,left=1.5cm,right=1.5cm,nohead,nofoot]{geometry}
%
\usepackage{xcolor}
\usepackage{enumitem}
\usepackage{amsmath,amsfonts,amssymb,mathtools}
\usepackage{tcolorbox}

\usepackage{tikz}
\usetikzlibrary{automata,shapes,arrows,positioning}
\tikzset{
 node distance=2.5cm,
 initial text={$M_{\mathcal{L}_\myling}$},
 double distance=1pt,
 every state/.style={semithick,fill=blue!20!white,minimum size=20pt,inner sep=0pt},
 every edge/.style={draw,->,>=stealth,auto,semithick,font=\ttfamily\small},
 accept/.style={accepting,fill=green!20!white},
 reject/.style={fill=red!20!white},
 slopea/.style={sloped,anchor=center,above},
 slopeb/.style={sloped,anchor=center,below}
}

%\newcommand{\concatL}{\ensuremath{{\scriptstyle\circ}}}%

\begin{document}
% ------------------------------------------------- Cabeçalho
 \begin{tcolorbox}[rounded corners, colback=blue!3, colframe=blue!40!black]
  \footnotesize\textbf{Universidade Federal de Goiás -- UFG}\hfill \textsc{Linguagens Formais e Autômatos -- 2022/2}\\
  \footnotesize\textbf{Instituto de Informática -- INF\hfill Prof. Humberto J. Longo} -- \scriptsize\texttt{longo@inf.ufg.br}
 \end{tcolorbox}\bigskip
%
% ------------------------------------------------- Atividade
\begin{tcolorbox}[rounded corners, colback=blue!2, colframe=blue!40!black, title=\textbf{Atividade AA-\ua\ -- \discente\ (\matricula)}]
 Nesta tarefa deve-se propor um autômato finito determinístico (DFA) resultante do produto $\otimes$ do DFA mínimo que reconhece a linguagem $\mathcal{L}_{par}$ (abaixo especificada) com um DFA que reconheça as cadeias da linguagem $\mathcal{L}$ selecionada. Especifique a tupla que define o DFA resultante da operação $\otimes$ e desenhe o correspondente diagrama de estados. (Cada aluno(a) deve consultar na descrição da atividade AA-\ua, na disciplina INF0333A da plataforma Turing, qual é a linguagem associada ao seu número de matrícula. A descrição da linguagem está disponível no arquivo ``Lista de linguagens regulares'' da Seção ``Coletânea de exercícios''.)
\end{tcolorbox}\bigskip
%
% ------------------------------------ Resolução do exercício
\begin{tcolorbox}[rounded corners, colback=yellow!5, colframe=red!40!black, title={$\mathcal{L}_{par}=\{w\in\{0,1\}^*\mid |w|=2\cdot k,\ k\in\mathbb{N}\}$}]
 \begin{itemize}[leftmargin=*]
%- - - - - - - - - - - - - - - - - - - - - - - - - - - - - - - -   
  \item Autômato finito determinístico que reconhece as cadeias pertencentes à linguagem $\mathcal{L}_{par}$:
   \begin{center}
   \begin{tikzpicture}[
    initial text={$M_{\mathcal{L}_{par}}$},
%    transform shape,
%    scale=1.2
   ]
    \node[state,initial,accept] (q0) {$q_0$};
    \node[state,right of=q0]    (q1) {$q_1$};
  %
    \draw (q0) edge[bend left=20] node {0,1} (q1)
          (q1) edge[bend left=20] node {0,1} (q0);
   \end{tikzpicture}
  \end{center}
 \end{itemize}
\end{tcolorbox}\bigskip
%=========================================================================
\begin{tcolorbox}[rounded corners, colback=yellow!5, colframe=red!40!black, title={$\mathcal{L}_\myling = \{w\mid w$ contém $010$ exatamente uma vez$\}$}]
 \begin{itemize}[leftmargin=*]
  \item $\mathcal{L}_\myling$ deve ser a linguagem associada ao número de matrícula de cada aluno.
%- - - - - - - - - - - - - - - - - - - - - - - - - - - - - - - - 
  \item Autômato finito determinístico que reconhece as cadeias da linguagem  $\mathcal{L}_\myling$ (o aconselhável é que seja o DFA mínimo):
  \begin{center}
   \begin{tikzpicture}[
%    node distance=2.cm,
%    transform shape,
%    scale=1.2
   ]
    \node[state,initial]            (s0) {$s_0$};
    \node[state,right of=s0]        (s1) {$s_1$};
    \node[state,right of=s1]        (s2) {$s_2$};
    \node[state,right of=s2,accept] (s3) {$s_3$};
    \node[state,right of=s3,accept] (s4) {$s_4$};
    \node[state,right of=s4,reject] (s5) {$s_5$};
  %
    \draw (s0) edge               node {0}   (s1)
               edge[loop above]   node {1}   (s0)
          (s1) edge[loop above]   node {0}   (s1)
               edge               node {1}   (s2)
          (s2) edge               node {0}   (s3)
               edge[bend left=30] node {1}   (s0)
          (s3) edge[loop above]   node {0}   (s3)
               edge[bend left=20] node {1}   (s4)
          (s4) edge[bend left=20] node {1}   (s3)
               edge               node {0}   (s5)
          (s5) edge[loop above]   node {0,1} (s5);
   \end{tikzpicture}
  \end{center}
 \end{itemize}
\end{tcolorbox}\bigskip
%=========================================================================
\begin{tcolorbox}[rounded corners, colback=yellow!5, colframe=red!40!black, title={$\mathcal{L}(M_{\mathcal{L}_{par}} \otimes M_{\mathcal{L}_\myling})  \equiv \mathcal{L}_{par} \cap \mathcal{L}_\myling \equiv \{w\mid |w| \text{ é par e } w \text{ contém 010 exatamente uma vez}\}.$}]
 \begin{itemize}[leftmargin=*]
%- - - - - - - - - - - - - - - - - - - - - - - - - - - - - - - - 
  \item Autômato finito determinístico que reconhece as cadeias da linguagem $\mathcal{L}(M_{\mathcal{L}_{par}}\otimes M_{\mathcal{L}_\myling})$:
  \begin{align*}
     D_{\otimes} & = \langle \Sigma, S,s_0{\cdot}q_0, \delta,F\rangle,\shortintertext{onde:}
     \Sigma & =\{0,1\},\\
     S & =\{s_0{\cdot}q_0,s_0{\cdot}q_1,s_1{\cdot}q_0,\allowbreak s_1{\cdot}q_1,s_2{\cdot}q_0,s_2{\cdot}q_1,s_3{\cdot}q_0,s_3{\cdot}q_1,s_4{\cdot}q_0,s_4{\cdot}q_1,s_5{\cdot}q_0,s_5{\cdot}q_1\},\\
     F & =\{s_3{\cdot}q_0,s_4{\cdot}q_0\},
  \end{align*}
  com a função $\delta$ definida por:
    $$\begin{array}{|c|cc|}
     \hline
     \delta        & 0             & 1\\
     \hline
     s_0{\cdot}q_0 & s_1{\cdot}q_1 & s_4{\cdot}q_1\\
     s_0{\cdot}q_1 & s_1{\cdot}q_0 & s_4{\cdot}q_0\\
     s_1{\cdot}q_0 & s_2{\cdot}q_1 & s_5{\cdot}q_1\\
     s_1{\cdot}q_1 & s_2{\cdot}q_0 & s_5{\cdot}q_0\\
     s_2{\cdot}q_0 & s_3{\cdot}q_1 & s_6{\cdot}q_1\\
     s_2{\cdot}q_1 & s_3{\cdot}q_0 & s_6{\cdot}q_0\\
     s_3{\cdot}q_0 & s_3{\cdot}q_1 & s_4{\cdot}q_1\\
     s_3{\cdot}q_1 & s_3{\cdot}q_0 & s_4{\cdot}q_0\\
     s_4{\cdot}q_0 & s_5{\cdot}q_1 & s_3{\cdot}q_1\\
     s_4{\cdot}q_1 & s_5{\cdot}q_0 & s_3{\cdot}q_0\\
     s_5{\cdot}q_0 & s_6{\cdot}q_1 & s_9{\cdot}q_1\\
     s_5{\cdot}q_1 & s_6{\cdot}q_0 & s_9{\cdot}q_0\\
     s_6{\cdot}q_0 & s_7{\cdot}q_1 & s_10{\cdot}q_1\\
     s_6{\cdot}q_1 & s_7{\cdot}q_0 & s_10{\cdot}q_0\\
     s_7{\cdot}q_0 & s_7{\cdot}q_1 & s_11{\cdot}q_1\\
     s_7{\cdot}q_1 & s_7{\cdot}q_0 & s_11{\cdot}q_0\\
     s_8{\cdot}q_0 & s_9{\cdot}q_1 & s_12{\cdot}q_1\\
     s_8{\cdot}q_1 & s_9{\cdot}q_0 & s_12{\cdot}q_0\\
     s_9{\cdot}q_0 & s_10{\cdot}q_1 & s_12{\cdot}q_1\\
     s_9{\cdot}q_1 & s_10{\cdot}q_0 & s_12{\cdot}q_0\\
     s_10{\cdot}q_0 & s_11{\cdot}q_1 & s_12{\cdot}q_1\\
     s_10{\cdot}q_1 & s_11{\cdot}q_0 & s_12{\cdot}q_0\\
     s_11{\cdot}q_0 & s_11{\cdot}q_1 & s_12{\cdot}q_1\\
     s_11{\cdot}q_1 & s_11{\cdot}q_0 & s_12{\cdot}q_0\\
     s_12{\cdot}q_0 & s_12{\cdot}q_1 & s_12{\cdot}q_1\\
     s_12{\cdot}q_1 & s_12{\cdot}q_0 & s_12{\cdot}q_0\\
     \hline
    \end{array}$$
 \end{itemize}
\end{tcolorbox}\bigskip
%=========================================================================
\begin{tcolorbox}[rounded corners, colback=yellow!5, colframe=red!40!black, title={$\mathcal{L}(M_{\mathcal{L}_{par}} \otimes M_{\mathcal{L}_\myling}) \equiv \mathcal{L}_{par} \cap \mathcal{L}_{\myling} \equiv \{w\mid |w| \text{ é par e } w \text{ contém 010 exatamente uma vez}\}.$}]
 \begin{itemize}[leftmargin=*]
%- - - - - - - - - - - - - - - - - - - - - - - - - - - - - - - - 
  \item Diagrama de estados do autômato finito determinístico $D_{\otimes}$:
  \begin{center}
   \begin{tikzpicture}[
    initial text={$D_{\otimes}$},
%    node distance=3.cm,
%    transform shape,
%    scale=0.8
   ]
    \node[state,initial above,ellipse]       (p00) {$s_0{\cdot}q_0$};
    \node[state,right of=p00,ellipse]        (p11) {$s_1{\cdot}q_1$};
    \node[state,above of=p11,ellipse]        (p20) {$s_2{\cdot}q_0$};
    \node[state,right of=p11,ellipse]        (p31) {$s_3{\cdot}q_1$};
    \node[state,right of=p31,ellipse,accept] (p40) {$s_4{\cdot}q_0$};
    \node[state,below of=p00,ellipse]        (p01) {$s_0{\cdot}q_1$};
    \node[state,right of=p01,ellipse]        (p10) {$s_1{\cdot}q_0$};
    \node[state,below of=p10,ellipse]        (p21) {$s_2{\cdot}q_1$};
    \node[state,right of=p10,ellipse,accept] (p30) {$s_3{\cdot}q_0$};
    \node[state,right of=p30,ellipse]        (p41) {$s_4{\cdot}q_1$};
    \node[state,right of=p40,ellipse,reject] (p51) {$s_5{\cdot}q_1$};
    \node[state,right of=p41,ellipse,reject] (p50) {$s_5{\cdot}q_0$};
   %
    \draw (p00) edge                                          node {0}   (p11)
                edge[bend left=20]                            node {1}   (p01)
          (p01) edge                                          node {0}   (p10)
                edge[bend left=20]                            node {1}   (p00)
          (p10) edge[bend left=20]                            node {0}   (p11)
                edge[left]                                    node {1}   (p21)
          (p11) edge[bend left=20]                            node {0}   (p10)
                edge                                          node {1}   (p20)
          (p20) edge[bend left=40]                            node {0}   (p31)
                edge[out=180,in=160,looseness=1.8,above left] node {1}   (p01)
%                edge[out=200,in=150,looseness=2.4,above left] node {1}   (p01)
          (p21) edge[bend right=40,below right]               node {0}   (p30)
                edge[out=180,in=200,looseness=1.8]            node {1}   (p00)
%                edge[out=160,in=210,looseness=2.4]            node {1}   (p00)
          (p30) edge[bend left=20]                            node {0}   (p31)
                edge[bend left=20]                            node {1}   (p41)
          (p31) edge[bend left=20]                            node {0}   (p30)
                edge[bend left=20]                            node {1}   (p40)
          (p40) edge[bend left=20]                            node {1}   (p31)
                edge                                          node {0}   (p51)
          (p41) edge[bend left=20]                            node {1}   (p30)
                edge                                          node {0}   (p50)
          (p50) edge[bend left=20]                            node {0,1} (p51)
%          (p50) edge[bend left=20,slopea]                     node {0,1} (p51)
          (p51) edge[bend left=20]                            node {0,1} (p50);
   \end{tikzpicture}
  \end{center}
 \end{itemize}
\end{tcolorbox}
%
%-----------------------------------------------
\end{document}
%
