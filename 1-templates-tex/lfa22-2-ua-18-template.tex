\documentclass[12pt]{article}

% ----------------------------------------- Dados do discente
% Insira os seus dados e do exercício escolhido:
\def\discente{Fulana(o) de tal}
\def\matricula{20010101}
\def\ua{18}
\def\myling{{99}} % Informe o número da linguagem selecionada.

% ------------------------------------------ Babel & Geometry
\usepackage[brazil]{babel}
\usepackage[T1]{fontenc}
\usepackage[utf8]{inputenc}
\usepackage[a4paper,top=1.0cm,bottom=2.0cm,left=1.5cm,right=1.5cm]{geometry}
%
\usepackage{xcolor}
\usepackage{enumitem}
\usepackage{amsmath,amsfonts,amssymb,mathtools}
\usepackage[breakable]{tcolorbox}
%
% -------------------------------------------------- New commands
%\newcommand{\tvs}{{\large\bf \textvisiblespace}}
\newcommand\vs[1][.8em]{% Visible space!
 \makebox[#1]{%
  \kern.07em
  \vrule height.5ex
  \hrulefill
  \vrule height.5ex
  \kern.07em
 }
}
\newcommand{\ve}{\ensuremath{\varepsilon}}

\begin{document}
% ------------------------------------------------- Cabeçalho
 \begin{tcolorbox}[rounded corners, colback=blue!3, colframe=blue!40!black]
  \footnotesize\textbf{Universidade Federal de Goiás -- UFG}\hfill \textsc{Linguagens Formais e Autômatos -- 2022/2}\\
  \footnotesize\textbf{Instituto de Informática -- INF\hfill Prof. Humberto J. Longo} -- \scriptsize\texttt{longo@inf.ufg.br}
 \end{tcolorbox}\bigskip
%
% ------------------------------------------------- Atividade
\begin{tcolorbox}[rounded corners, colback=blue!2, colframe=blue!40!black, title=\textbf{Atividade AA-\ua}]
 Nesta tarefa deve-se selecionar uma das linguagens listadas na descrição da AA-\ua na disciplina INF0333A da plataforma Turing e demonstrar formalmente (com o auxílio do \emph{Pumping Lemma} para linguagens livres de contexto) que a linguagem escolhida não é livre de contexto. A descrição da linguagem está disponível no arquivo ``Lista de linguagens que não são livres de contexto'' (vide Seção ``Coletânea de exercícios'', na disciplina INF0333A da plataforma Turing).
\end{tcolorbox}\bigskip

%
% ------------------------------------ Resolução do exercício
%=========================================================================
\begin{tcolorbox}[rounded corners, colback=yellow!5, colframe=red!40!black, title={\discente\ (\matricula)}]
 $\mathcal{L}_{\myling} = \{w\in\{0,1,2\}^*\mid w = 0^n1^n2^n,\ n \geqslant 0\}$.
\end{tcolorbox}
%=========================================================================
\begin{tcolorbox}[rounded corners, breakable, colback=yellow!5, colframe=red!40!black, title={A linguagem $\mathcal{L}_{\myling}$ não é livre de contexto.}]
Suponha que $\mathcal{L}_{\myling}$ seja livre de contexto. O \emph{Pumping Lemma} para linguagens livres de contexto garante a existência de $p\in\mathbb{Z}^+$ (\emph{pumping length}), tal que qualquer cadeia $w\in \mathcal{L}_{\myling}$, com $|w|\geqslant p$, pode ser subdividida em subcadeias $u$, $v$, $x$, $y$ e $z$ ($w=uvxyz$) satisfazendo $|vxy|\leqslant p$, $|vy|>0$ ($v\neq\varepsilon$ e/ou $y\neq\varepsilon$) e $uv^ixy^iz\in \mathcal{L}_{\myling}$, para $i\geqslant 0$.\\[10pt]
% - - - - - - - - - - - - - - - - 
Contudo, considere a cadeia $w=uvxyz=0^p1^p2^p\in \mathcal{L}_{\myling}$. Como o \emph{Pumping Lemma} especifica que $|vxy|\leqslant p$, então os três símbolos do alfabeto $\Sigma$ não podem ocorrer ao mesmo tempo em cada uma das subcadeias $v$ e $y$. Assim, se $v$ e $y$ contém, cada um, apenas um dos símbolos de $\Sigma$, então $w'=uv^2xy^2z\notin \mathcal{L}_{\myling}$, pois $w'$ conterá menos 0's do que 1's ou 2's ou menos 2's do que 0's ou 1's. Se $v$ ou $y$ contém mais de um dos símbolos de $\Sigma$, novamente $w'=uv^2xy^2z\notin \mathcal{L}_{\myling}$, pois $w'$ conterá sequências de 0's e 1's intercalados ou de 1's e 2's intercalados.\\[10pt]
% - - - - - - - - - - - - - - - - 
 Logo, dadas as contradições ao \emph{Pumping Lemma}, é falsa a suposição de que $\mathcal{L}_{\myling}$ é livre de contexto.
\end{tcolorbox}
%=========================================================================
\end{document}
%
