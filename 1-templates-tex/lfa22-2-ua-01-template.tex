\documentclass[12pt]{article}

% ----------------------------------------- Dados do discente
% Insira os seus dados e do exercício escolhido:
\def\discente{Fulana(o) de tal}
\def\matricula{20010101}
\def\ua{01}
\def\myling{{99}} % Informe o número da linguagem selecionada.

% ------------------------------------------ Babel & Geometry
\usepackage[brazil]{babel}
\usepackage[T1]{fontenc}
\usepackage[utf8]{inputenc}
\usepackage[a4paper,top=0.5cm,bottom=1.5cm,left=1.5cm,right=1.5cm,nohead,nofoot]{geometry}
%
\usepackage{xcolor}
\usepackage{enumitem}
\usepackage{mathtools}
\usepackage{tcolorbox}

\begin{document}
% ------------------------------------------------- Cabeçalho
 \begin{tcolorbox}[rounded corners, colback=blue!3, colframe=blue!40!black]
  \footnotesize\textbf{Universidade Federal de Goiás -- UFG}\hfill \textsc{Linguagens Formais e Autômatos -- 2022/2}\\
  \footnotesize\textbf{Instituto de Informática -- INF\hfill Prof. Humberto J. Longo} -- \scriptsize\texttt{longo@inf.ufg.br}
 \end{tcolorbox}\bigskip
%
% ------------------------------------------------- Atividade
\begin{tcolorbox}[rounded corners, colback=blue!2, colframe=blue!40!black, title=\textbf{Atividade AA-\ua}]
  Nesta tarefa deve-se propôr uma definição recursiva para a linguagem selecionada. (Cada aluna(o) deve consultar na descrição da atividade AA-\ua, na disciplina INF0333A da plataforma Turing, qual é a linguagem associada ao seu número de matrícula. A descrição da linguagem está disponível no arquivo ``Lista de linguagens regulares'' da Seção ``Coletânea de exercícios''.)
\end{tcolorbox}\bigskip
%
% ------------------------------------ Resolução do exercício
\begin{tcolorbox}[rounded corners, colback=yellow!5, colframe=red!40!black, title=\textbf{\matricula\ -- \discente}]
 \begin{itemize}[leftmargin=*]
% 
  \item $\mathcal{L}_\myling = \{w\in\Sigma^*\mid |w|$ é par e $w$ começa com 1$; \Sigma=\{0,1\}\}$.
%
  \item  Definição recursiva de $\mathcal{L}_\myling$:
%
  \begin{description}[labelwidth=\widthof{Recursão},labelindent=0.25\labelwidth,leftmargin=1.4\labelwidth,align=right]
   \item [Base:] $10,11 \in \mathcal{L}$.
   \item [Recursão:] Se $u\in \mathcal{L}$, então $u00,u01,u10,u11\in\mathcal{L}$.
   \item [Fecho:] Dada uma cadeia $u\in\Sigma^*$, $u\in\mathcal{L}_\myling$ se pode ser obtida a partir das cadeias básicas, com a aplicação da regra recursiva um número finito de vezes.
  \end{description}  
% 
 \end{itemize}
\end{tcolorbox}
%
%-----------------------------------------------
\end{document}
%
