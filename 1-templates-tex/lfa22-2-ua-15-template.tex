\documentclass[12pt]{article}

% ----------------------------------------- Dados do discente
% Insira os seus dados e do exercício escolhido:
\def\discente{Fulana(o) de tal}
\def\matricula{20010101}
\def\ua{15}
\def\up{14}
\def\myling{{99}} % Informe o número da linguagem selecionada.

% ------------------------------------------ Babel & Geometry
\usepackage[brazil]{babel}
\usepackage[T1]{fontenc}
\usepackage[utf8]{inputenc}
\usepackage[a4paper,top=1.0cm,bottom=2.0cm,left=1.5cm,right=1.5cm]{geometry}
%
\usepackage{xcolor}
\usepackage{enumitem}
\usepackage{amsmath,amsfonts,amssymb,mathtools}
\usepackage[breakable]{tcolorbox}
%
\newcommand{\ve}{\ensuremath{\varepsilon}}

\begin{document}
% ------------------------------------------------- Cabeçalho
 \begin{tcolorbox}[rounded corners, colback=blue!3, colframe=blue!40!black]
  \footnotesize\textbf{Universidade Federal de Goiás -- UFG}\hfill \textsc{Linguagens Formais e Autômatos -- 2022/2}\\
  \footnotesize\textbf{Instituto de Informática -- INF\hfill Prof. Humberto J. Longo} -- \scriptsize\texttt{longo@inf.ufg.br}
 \end{tcolorbox}\bigskip
%
% ------------------------------------------------- Atividade
\begin{tcolorbox}[rounded corners, colback=blue!2, colframe=blue!40!black, title=\textbf{Atividade AA-\ua}]
 Nesta tarefa deve-se converter a gramática $G_n^5$, obtida na atividade avaliativa anterior (AA-\up\ relativa a transformações em GLC's), para a forma normal de Chomsky \textbf{OU} para a forma normal de Greibach. (Cada aluno(a) deve consultar na descrição da atividade AA--\up, na disciplina INF0333A da plataforma Turing, qual é a linguagem associada ao seu número de matrícula. A descrição da linguagem está disponível no arquivo ``Lista de linguagens livres de contexto'' da Seção ``Coletânea de exercícios''.)
\end{tcolorbox}\bigskip

%
% ------------------------------------ Resolução do exercício
%=========================================================================
\begin{tcolorbox}[rounded corners, colback=yellow!5, colframe=red!40!black, title={\discente\ (\matricula)}]
\begin{itemize}
%- - - - - - - - - - - - - - - - - - - - - - - - - - - - - - - 
  \item  $\mathcal{L}_{\myling} = ???$
%- - - - - - - - - - - - - - - - - - - - - - - - - - - - - - - - 
  \item Gramática $G_{\myling}^5$ (obtida na AA-\up) que gera as cadeias da linguagem $\mathcal{L}_{\myling}$:\\
  $G_{\myling}^5=(V,\Sigma,P,S_0)=(\{A,B,S,S_0,R\},\{a,b\},P,S_0)$, com
    \[P=\left\{
     \begin{aligned}
      S_0 & \to SA\mid aB\mid a, \\
      S   & \to aB\mid a\mid aBR\mid aR, \\
      R   & \to AR\mid b\mid SA\mid aB\mid a, \\
      A   & \to b\mid SA\mid aB\mid a, \\
      B   & \to b
     \end{aligned}
    \right\}\]
%- - - - - - - - - - - - - - - - - - - - - - - - - - - - - - - 
 \end{itemize}
\end{tcolorbox}\bigskip

%=========================================================================
\begin{tcolorbox}[breakable,rounded corners, colback=yellow!5, colframe=red!40!black, title={Forma normal de Chomsky.}]
%-----------------------------------------------
\begin{itemize}
  \item Gramática $G_{\myling}^6$ na forma normal de Chomsky, obtida a partir de $G_{\myling}^5$, que gera as cadeias da linguagem $\mathcal{L}_{\myling}$:\\
  $G_{\myling}^6=(V,\Sigma,P,S_0)=(\{A,B,S,S_0,R,T_1,T_2\},\{a,b\},P,S_0)$, com
    \[P=\left\{
     \begin{aligned}
      S_0 & \to SA\mid T_1B\mid a, \\
      S   & \to T_1B\mid a\mid T_1T_2\mid T_1R, \\
      R   & \to AR\mid b\mid SA\mid T_1B\mid a, \\
      A   & \to b\mid SA\mid T_1B\mid a, \\
      B   & \to b,\\
      T_1 & \to a,\\
      T_2 & \to BR
     \end{aligned}
    \right\}\]
\end{itemize}
\end{tcolorbox}

%=========================================================================
\begin{tcolorbox}[breakable,rounded corners, colback=yellow!5, colframe=red!40!black, title={Forma normal de Greibach.}]
%-----------------------------------------------
\begin{itemize}
  \item Gramática intermediária $G_{\myling}^7$, obtida de $G_{\myling}^6$, definindo-se números de ordem para as variáveis e fazendo-se as substituições apropriadas nas regras de derivação das variáveis $S_0$, $A$, $R$ e $T_2$:
      $$
       \begin{array}{l|ccccccccccc}
        \textbf{Variável} & S_0 & S & A & B & R & T_1 & T_2\\
        \hline
        \textbf{Ordem}    &   1 & 2 & 3 & 4 & 5 & 6   & 7\\
       \end{array}
      $$
  
  \noindent
  $G_{\myling}^7=(V,\Sigma,P,S_0)=(\{A,B,S_0,R,T_1,T_2\},\{a,b\},P,S_0)$, com
    \[P=\left\{
     \begin{aligned}
      S_0 & \to T_1BA\mid aA\mid T_1T_2A\mid T_1RA\mid T_1B\mid a, \\
      R   & \to bR\mid T_1BAR\mid aAR\mid T_1T_2AR\mid T_1RAR\mid T_1BR\mid\\
          & \qquad aR\mid b\mid T_1BA\mid aA\mid T_1T_2A\mid T_1RA\mid T_1B\mid a, \\
      A   & \to b\mid T_1BA\mid aA\mid T_1T_2A\mid T_1RA\mid T_1B\mid a, \\
      B   & \to b,\\
      T_1 & \to a,\\
      T_2 & \to bR
     \end{aligned}
    \right\}\]
%- - - - - - - - - - - - - - - - - - - - - - - - - - - - - - - -
  \item Gramática $G_{\myling}^8$ na forma normal de Greibach, obtida de $G_{\myling}^7$, substituindo-se em $G_{\myling}^7$ a variável $T_1$ mais à esquerda nas regras de derivação e eliminando-se as variáveis $S$ e $T_1$:\\
  $G_{\myling}^8=(V,\Sigma,P,S_0)=(\{A,B,S_0,R,T_2\},\{a,b\},P,S_0)$, com
    \[P=\left\{
     \begin{aligned}
      S_0 & \to aBA\mid aA\mid aT_2A\mid aRA\mid aB\mid a, \\
      R   & \to bR\mid aBAR\mid aAR\mid aT_2AR\mid aRAR\mid aBR\mid\\
          & \qquad aR\mid b\mid aBA\mid aA\mid aT_2A\mid aRA\mid aB\mid a, \\
      A   & \to b\mid aBA\mid aA\mid aT_2A\mid aRA\mid aB\mid a, \\
      B   & \to b,\\
      T_2 & \to bR
     \end{aligned}
    \right\}\]
\end{itemize}
\end{tcolorbox}
%=========================================================================
\end{document}
%
