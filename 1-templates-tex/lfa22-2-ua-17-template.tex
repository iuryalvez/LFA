\documentclass[12pt]{article}

% ----------------------------------------- Dados do discente
% Insira os seus dados e do exercício escolhido:
\def\discente{Fulana(o) de tal}
\def\matricula{20010101}
\def\ua{17}
\def\up{16}
\def\myling{{99}} % Informe o número da linguagem selecionada.

% ------------------------------------------ Babel & Geometry
\usepackage[brazil]{babel}
\usepackage[T1]{fontenc}
\usepackage[utf8]{inputenc}
\usepackage[a4paper,top=1.0cm,bottom=2.0cm,left=1.5cm,right=1.5cm]{geometry}
%
\usepackage{marvosym} % \Smiley
\usepackage{xcolor}
\usepackage{enumitem}
\usepackage{amsmath,amsfonts,amssymb,mathtools}
\usepackage[breakable]{tcolorbox}
%
\usepackage{tikz}
\usetikzlibrary{automata,arrows,positioning,shapes}
\tikzset{
 node distance=3cm,
 initial text={$P_{\myling}$},
 double distance=1pt,
 every state/.style={semithick,fill=blue!20!white,minimum size=20pt,inner sep=0pt},
 every edge/.style={draw,->,>=stealth,auto,semithick,font=\ttfamily\small}
}
%
% -------------------------------------------------- New commands
%\newcommand{\tvs}{{\large\bf \textvisiblespace}}
\newcommand\vs[1][.8em]{% Visible space!
 \makebox[#1]{%
  \kern.07em
  \vrule height.5ex
  \hrulefill
  \vrule height.5ex
  \kern.07em
 }
}
\newcommand{\ve}{\ensuremath{\varepsilon}}

\begin{document}
% ------------------------------------------------- Cabeçalho
 \begin{tcolorbox}[rounded corners, colback=blue!3, colframe=blue!40!black]
  \footnotesize\textbf{Universidade Federal de Goiás -- UFG}\hfill \textsc{Linguagens Formais e Autômatos -- 2022/2}\\
  \footnotesize\textbf{Instituto de Informática -- INF\hfill Prof. Humberto J. Longo} -- \scriptsize\texttt{longo@inf.ufg.br}
 \end{tcolorbox}\bigskip
%
% ------------------------------------------------- Atividade
\begin{tcolorbox}[rounded corners, colback=blue!2, colframe=blue!40!black, title=\textbf{Atividade AA-\ua}]
 Nesta tarefa deve ser escolhida \textbf{uma} das seguintes opções:
 \begin{enumerate}
  \item Propor uma gramática livre de contexto $G_n$ que gere as cadeias da linguagem selecionada e construir, segundo um dos algoritmos apresentados nas aulas, um PDA que reconheça as cadeias da linguagem gerada pela gramática.
  \item A partir do PDA $P_n$ obtido na atividade ``AA-16 : Autômatos com pilha'', obter uma gramática que gere a linguagem aceita pelo PDA.
 \end{enumerate}
 (Cada aluna(o) deve consultar na descrição da atividade AA--\up, na disciplina INF0333A da plataforma Turing, qual é a linguagem associada ao seu número de matrícula. A descrição da linguagem está disponível no arquivo ``Lista de linguagens livres de contexto'' da Seção ``Coletânea de exercícios''.)
\end{tcolorbox}\bigskip

%
% ------------------------------------ Resolução do exercício
%=========================================================================
\begin{tcolorbox}[rounded corners, colback=yellow!5, colframe=red!40!black, title={\discente\ (\matricula)}]
\begin{itemize}%[before=\color{blue}]
  \item $\mathcal{L}_{\myling} = \{a^nba^n \mid n \geqslant 0\}$.
%- - - - - - - - - - - - - - - - - - - - - - - - - - - - - - - - 
  \item Gramática $G_{\myling}$ que gera as cadeias da linguagem $\mathcal{L}_{\myling}$:\\ $G_{\myling}=(V,\Sigma,P,S)=(\{S,B\},\{a,b\},P,S)$, com
  $
   P =
   \left\{\begin{array}{l}
    S \to aSa \mid B,\\
    B \to b
   \end{array}\right\}.
  $
%- - - - - - - - - - - - - - - - - - - - - - - - - - - - - - - - 
  \item PDA $P_{\myling}$ que reconhece as cadeias da linguagem $\mathcal{L}_{\myling}$:
   \begin{center}
   \begin{tikzpicture}[
     accept/.style={accepting,fill=green!20!white},
     initial text={$P_\myling$},
   %  transform shape,
   %  scale=0.95
    ]
     \node[initial,state]            (s0) {$s_0$};
     \node[state,accept,right of=s0] (s1) {$s_1$};
   
     \path (s0) edge [loop above] node {$a,\ve\to X$}   (s0)
                edge              node {$b,\ve\to \ve$} (s1)
           (s1) edge [loop above] node {$a,X\to \ve$}   (s1);
   \end{tikzpicture}
   \end{center}
\end{itemize}
\end{tcolorbox}
%=========================================================================
\begin{tcolorbox}[rounded corners, colback=yellow!5, colframe=red!40!black, title={OPÇÂO 1(a): PDA construído a partir da gramática $G_{\myling}$}]
\begin{itemize}
 \item PDA $P^1_{\myling}$, construído a partir da gramática $G_{\myling}$, que reconhece as cadeias da linguagem $\mathcal{L}_{\myling}$:
   \begin{center}
   \begin{tikzpicture}[
     accept/.style={accepting,fill=green!20!white},
     slopea/.style={sloped,anchor=center,above},
     slopeb/.style={sloped,anchor=center,below},
     initial text={$P_\myling^1$},
   % transform shape,
   %  scale=0.95
    ]
     \node[initial,state]                  (s0) {$s_0$};
     \node[state,right of=s0]              (s1) {$s_1$};
     \node[state,right of=s1]              (s2) {$s_2$};
%     \node[state,below left of=s2,xshift=20pt]   (s3) {$s_3$};
%     \node[state,below right of=s2,xshift=-20pt] (s4) {$s_4$};
     \node[state,below of=s2,xshift=-50pt] (s3) {$s_3$};
     \node[state,below of=s2,xshift=50pt]  (s4) {$s_4$};
     \node[state,accept,right of=s2]       (s5) {$s_5$};
   
     \path (s0) edge node {$\ve,\ve\to \$$} (s1)
           (s1) edge node {$\ve,\ve\to S$}  (s2)
           (s2) edge [bend right=15,slopea]
                     node {$\ve,S\to a$}    (s3)
                edge node {$\ve,\$\to \ve$} (s5)
%                edge [loop above,align=center]
%                     node {$\ve,S\to B$\\[-2pt]$\ve,B\to b$\\[-2pt]
%                           $a,a\to \ve$\\[-2pt]$b,b\to \ve$}  (s2)
                edge [loop above,align=center]
                     node {$\begin{aligned}
                             \ve,S &\to B\\[-5pt]
                             \ve,B &\to b\\[-5pt]
                             a,a &\to \ve\\[-5pt]
                             b,b &\to \ve\\[-5pt]
                            \end{aligned}$}  (s2)
           (s3) edge [bend right=15,slopeb]
                     node {$\ve,\ve\to S$}  (s4)
           (s4) edge [bend right=15,slopea]
                     node {$\ve,\ve\to a$}  (s2);
   \end{tikzpicture}
   \end{center}
 \end{itemize}
\end{tcolorbox}
%=========================================================================
\begin{tcolorbox}[rounded corners, colback=yellow!5, colframe=red!40!black, title={OPÇÂO 1(b): PDA construído a partir da forma normal de Greibach da gramática $G_{\myling}$}]
\begin{itemize}
 \item Versão da gramática $G_{\myling}$ na forma normal de Greibach:\\ $G^1_{\myling}=(V,\Sigma,P,S)=(\{S_0,S,A\},\{a,b\},P,S)$, com
    $
     P =
     \left\{\begin{array}{r@{\;\to\;}l}
      S_0 & aSA \mid b,\\
      S   & aSA \mid b,\\
      A   & a
     \end{array}\right\}.
    $
%- - - - - - - - - - - - - - - - - - - - - - - - - - - - - - - -
 \item PDA $P^2_{\myling}$ obtido a partir da gramática $G^1_{\myling}$:
     \begin{center}
     \begin{tikzpicture}[
       accept/.style={accepting,fill=green!20!white},
       slopea/.style={sloped,anchor=center,above},
       slopeb/.style={sloped,anchor=center,below},
       initial text={$P^2_\myling$},
     %  transform shape,
     %  scale=0.95
      ]
       \node[initial,state]            (s0) {$s_0$};
       \node[state,right of=s0]        (s1) {$s_1$};
       \node[state,right of=s1]        (s2) {$s_2$};
       \node[state,accept,right of=s2] (s3) {$s_3$};
     
       \path (s0) edge node {$\ve,\ve\to \$$}       (s1)
             (s1) edge [align=left]
                       node {$b,\ve\to \ve$\\[-2pt]
                             $a,\ve\to SA$}         (s2)
             (s2) edge [loop above,align=left,pos=.7]
                       node {$a,A\to \ve$\\[-2pt]
                             $b,S\to \ve$\\[-2pt]
                             $a,S\to SA$}           (s2)
             (s2) edge node {$\ve,\$\to \ve$}       (s3);
     \end{tikzpicture}
     \end{center}
 \end{itemize}
\end{tcolorbox}
%=========================================================================
\begin{tcolorbox}[rounded corners, colback=yellow!5, colframe=red!40!black, title={OPÇÂO 2: Gramática $G^2_{\myling}$ obtida a partir do PDA $P_{\myling}$}]
 \begin{itemize}
  \item Vide exemplo 7.28 nos \emph{slides} \ldots {\Large \Smiley}
 \end{itemize}
\end{tcolorbox}
%=========================================================================
\end{document}
%
