\documentclass[12pt]{article}

% ----------------------------------------- Dados do discente
% Insira os seus dados e do exercício escolhido:
\def\discente{Fulana(o) de tal}
\def\matricula{20010101}
\def\ua{08}
\def\myling{{99}} % Informe o número da linguagem selecionada.

% ------------------------------------------ Babel & Geometry
\usepackage[brazil]{babel}
\usepackage[T1]{fontenc}
\usepackage[utf8]{inputenc}
\usepackage[a4paper,top=0.5cm,bottom=1.5cm,left=1.5cm,right=1.5cm,nohead,nofoot]{geometry}
%
\usepackage{xcolor}
\usepackage{enumitem}
\usepackage{amsmath,amsfonts,amssymb,mathtools}
\usepackage[breakable]{tcolorbox}
\usepackage{booktabs}

\usepackage{tikz}
\usetikzlibrary{automata,shapes,arrows,positioning}
\tikzset{
 node distance=2.5cm,
 initial text={$M_{\mathcal{L}_\myling}$},
 double distance=1pt,
 every state/.style={semithick,fill=blue!20!white,minimum size=20pt,inner sep=0pt},
 every edge/.style={draw,->,>=stealth,auto,semithick,font=\ttfamily\small},
 accept/.style={accepting,fill=green!20!white},
 reject/.style={fill=red!20!white},
 slopea/.style={sloped,anchor=center,above},
 slopeb/.style={sloped,anchor=center,below}
}

%=-=-=-=-=-=-=-=-=-=-=-=-=-=-=-=-=-= N e w   c o m m a n d s
\newcommand{\ve}{\ensuremath{\varepsilon}}

\begin{document}
% ------------------------------------------------- Cabeçalho
 \begin{tcolorbox}[rounded corners, colback=blue!3, colframe=blue!40!black]
  \footnotesize\textbf{Universidade Federal de Goiás -- UFG}\hfill \textsc{Linguagens Formais e Autômatos -- 2022/2}\\
  \footnotesize\textbf{Instituto de Informática -- INF\hfill Prof. Humberto J. Longo} -- \scriptsize\texttt{longo@inf.ufg.br}
 \end{tcolorbox}\bigskip
%
% ------------------------------------------------- Atividade
\begin{tcolorbox}[rounded corners, colback=blue!2, colframe=blue!40!black, title=\textbf{Atividade AA-\ua}]
  Nesta tarefa deve-se ($i$) propôr um autômato finito determinístico \textbf{mínimo} $D$ que reconheça as cadeias da linguagem selecionada, a partir de $D$ construir uma gramática $G_1$ que gere as cadeias reconhecidas por $D$ e a partir de $G_1$, usando o algoritmo baseado em sistemas de equações, extrair uma expressão regular $\mathcal{R}_1$ que gere as mesmas cadeias geradas por $G_1$; ($ii$) propôr um autômato finito não-determinístico $N$ que reconheça as cadeias da linguagem selecionada e, como no item ($i$), obter a partir de $N$ uma gramática $G_2$ (não necessariamente será regular!) e, a partir desta, obter uma expressão regular $\mathcal{R}_2$. O autômato $N$ pode ser um NFA ou NFA-$\varepsilon$, com pelo menos uma transição não determinística ou uma transição $\varepsilon$. \textbf{Atenção:} NFA's criados a partir do simples acréscimo de transições $\delta(s_i,\varepsilon)=s_i$ ($\varepsilon$-laços) a um DFA não serão considerados corretos, por não permitirem uma avaliação razoável do aprendizado dos conceitos abordados nesta atividade avaliativa. (Cada aluna(o) deve consultar na descrição da atividade AA--\ua, na disciplina INF0333A da plataforma Turing, qual é a linguagem associada ao seu número de matrícula. A descrição da linguagem está disponível no arquivo ``lista de linguagens regulares'' da Seção ``Coletânea de exercícios''.)
\end{tcolorbox}\bigskip

%
% ------------------------------------ Resolução do exercício
\begin{tcolorbox}[rounded corners, colback=yellow!5, colframe=red!40!black, title={\discente\ (\matricula)}]
 \begin{itemize}[leftmargin=*]
% - - - - - - - - - - - - - - - - - - - - - - - - - - - - - - - - - - - -
  \item $\mathcal{L}_\myling = \{w\mid w$ contém $010$ exatamente uma vez$\}$
% - - - - - - - - - - - - - - - - - - - - - - - - - - - - - - - - - - - -
  \item $ER(\mathcal{L}_\myling) = (1\cup 0^+11)^*0^+10(0\cup 11)^*(1\cup\varepsilon)$.
% - - - - - - - - - - - - - - - - - - - - - - - - - - - - - - - - - - - -
 \end{itemize}
\end{tcolorbox}\bigskip
%=========================================================================
\begin{tcolorbox}[breakable,rounded corners, colback=yellow!5, colframe=red!40!black, title={DFA mínimo que reconhece as cadeias de $\mathcal{L}_\myling$}]
 \centering
 \begin{tikzpicture}[
%    transform shape,
%    scale=0.8
 ]
  \node[state,initial]            (s0) {$S$};
  \node[state,right of=s0]        (s1) {$A$};
  \node[state,right of=s1]        (s2) {$B$};
  \node[state,right of=s2,accept] (s3) {$C$};
  \node[state,right of=s3,accept] (s4) {$D$};
  \node[state,right of=s4,reject] (s5) {$s_5$};
%
  \draw (s0) edge               node {0}   (s1)
             edge[loop above]   node {1}   (s0)
        (s1) edge[loop above]   node {0}   (s1)
             edge               node {1}   (s2)
        (s2) edge               node {0}   (s3)
             edge[bend left=30] node {1}   (s0)
        (s3) edge[loop above]   node {0}   (s3)
             edge[bend left=20] node {1}   (s4)
        (s4) edge[bend left=20] node {1}   (s3)
             edge               node {0}   (s5)
        (s5) edge[loop above]   node {0,1} (s5);
 \end{tikzpicture}
\end{tcolorbox}\bigskip
%=========================================================================
\begin{tcolorbox}[breakable,rounded corners, colback=yellow!5, colframe=red!40!black, title={Gramática $G_1$ que gera as cadeias de $\mathcal{L}_\myling$}]
 \begin{align*}
 G_1 &= (V,\Sigma,P,S)=(\{A,B,C,D,S\},\{0,1\},P,S),\text{ com:}\\
   P &=
   \left\{\begin{array}{l}
    S \to 0A \mid 1S,\\
    A \to 0A \mid 1B,\\
    B \to 0C \mid 1S,\\
    C \to 0C \mid 1D \mid \varepsilon,\\
    D \to  1C \mid \varepsilon
   \end{array}\right\}.
 \end{align*}
\end{tcolorbox}\bigskip
%=========================================================================
\begin{tcolorbox}[breakable,rounded corners, colback=yellow!5, colframe=red!40!black, title={Extração de expressão regular $\mathcal{R}_1$ da gramática $G$, tal que $\mathcal{L}(\mathcal{R}_1)=\mathcal{L}(G_1)$}]
  $$
  \begin{array}{ccll}
  \toprule
  \textbf{Etapa} & \textbf{Eq.} & \textbf{Expressão} & \textbf{Ação}\\
  \midrule
    I & 1 & S = 0A \cup 1S           &\\
      & 2 & A = 0A \cup 1B           &\\
      & 3 & B = 0C \cup 1S           &\\
      & 4 & C = 0C \cup 1D  \cup \ve &\\
      & 5 & D = 1C \cup \ve          &\\
  \midrule
   II & 1 & S = 0A \cup 1S                 &\\
      & 2 & A = 0^*1B                      & I.2 \to \texttt{Lema de Arden}\\
      & 3 & B = 0C \cup 1S                 &\\
      & 4 & C = (0\cup 11)C \cup 1\cup \ve & I.5 \to I.4, \texttt{Fatoração}\\
  \midrule
  III & 1 & S = 0A \cup 1S              &\\
      & 2 & A = 0^*10C\cup 0^*11S       & II.3 \to II.2\\
      & 3 & C = (0\cup 11)^*(1\cup \ve) & \texttt{Lema de Arden}\\
  \midrule
   IV & 1 & S = 0A \cup 1S                              &\\
      & 2 & A = 0^*10(0\cup 11)^*(1\cup \ve)\cup 0^*11S & III.3 \to III.2\\
  \midrule
    V & 1 & S = 0^+10(0\cup 11)^*(1\cup \ve)\cup 0^+11S \cup 1S & IV.2 \to IV.1\\
  \midrule
   VI & 1 & S = (1\cup 0^+11)^*0^+10(0\cup 11)^*(1\cup \ve) & \text{Fatoração, Lema de Arden}\\
  \bottomrule
  \end{array}
  $$
\end{tcolorbox}\bigskip
%=========================================================================
\begin{tcolorbox}[breakable,rounded corners, colback=yellow!5, colframe=red!40!black, title={NFA que reconhece as cadeias de $\mathcal{L}_\myling$}]
 \centering
 \begin{tikzpicture}
  [
   initial text={$N_\myling$},
   accept/.style={accepting,fill=green!20!white},
   reject/.style={fill=red!20!white},
%    node distance=2.cm,
%    transform shape,
%    scale=1.2
  ]
    \node[state,initial]                            (s) {$S$};0
    \node[state,below left=1cm of s,yshift=-1.2cm]  (a) {$A$};1
    \node[state,below right=1cm of s,yshift=-1.2cm] (b) {$B$};2
    \node[state,right of=s]                         (c) {$C$};3
    \node[state,right of=c]                         (d) {$D$};4
    \node[state,right of=d]                         (e) {$E$};5
    \node[state,right of=e]                         (f) {$F$};6
    \node[state,below of=f]                         (g) {$G$};7
    \node[state,right of=f,accept]                  (h) {$H$};8
  %
    \draw (s) edge [bend right=20,left]  node {0}               (a)
              edge                       node {$\varepsilon$}   (c)
              edge [loop above]          node {1}               (s)
          (a) edge [loop below]          node {0}               (a)
              edge [bend right=20,below] node {1}               (b)
          (b) edge [bend right=20,right] node {1}               (s)
          (c) edge [loop above]          node {0}               (c)
              edge                       node {0}               (d)
          (d) edge                       node {1}               (e)
          (e) edge                       node {0}               (f)
          (f) edge [loop above]          node {0}               (f)
              edge [bend left=20]        node {1}               (g)
              edge                       node {$\varepsilon,1$} (h)
          (g) edge [bend left=20]        node {1}               (f);
   \end{tikzpicture}
\end{tcolorbox}\bigskip
%=========================================================================
\begin{tcolorbox}[breakable,rounded corners, colback=yellow!5, colframe=red!40!black, title={Gramática $G_2$ que gera as cadeias da linguagem $\mathcal{L}_{\myling}$}]
 \begin{align*}
 G_2 &=(V,\Sigma,P,S)=(\{A,B,C,D,E,F,G,H,S\},\{0,1\},P,S), \text{ com}\\
   P &=
   \left\{\begin{array}{l}
    S \to 0A \mid \varepsilon C \mid 1S,\\
    A \to 0A \mid 1B,\\
    B \to 1S\\
    C \to 0C \mid 0D,\\
    D \to 1E\\
    E \to 0F,\\
    F \to 0F \mid 1G \mid 1H \mid \varepsilon H\\
    G \to 1F,\\
    H \to \varepsilon
   \end{array}\right\}
   =
   \left\{\begin{array}{l}
    S \to 0A \mid C \mid 1S,\\
    A \to 0A \mid 1B,\\
    B \to 1S\\
    C \to 0C \mid 0D,\\
    D \to 1E\\
    E \to 0F,\\
    F \to 0F \mid 1G \mid 1H \mid H\\
    G \to 1F,\\
    H \to \varepsilon
   \end{array}\right\}.
 \end{align*}
\end{tcolorbox}\bigskip
%%=========================================================================
\begin{tcolorbox}[breakable,rounded corners, colback=yellow!5, colframe=red!40!black, title={Extração de expressão regular $\mathcal{R}_2$ da gramática $G_2$, tal que $\mathcal{L}(\mathcal{R}_2)=\mathcal{L}(G_2)$}]
  $$
  \begin{array}{ccll}
  \toprule
  \textbf{Etapa} & \textbf{Eq.} & \textbf{Expressão} & \textbf{Ação}\\
  \midrule
    I & 1 & S = 0A \cup C \cup 1S        &\\
      & 2 & A = 0A \cup 1B               &\\
      & 3 & B = 1S                       &\\
      & 4 & C = 0C \cup 0D               &\\
      & 5 & D = 1E                       &\\
      & 6 & E = 0F                       &\\
      & 7 & F = 0F \cup 1G \cup 1H \cup H&\\
      & 8 & G = 1F                       &\\
      & 9 & H = \ve                      &\\
  \midrule
   II & 1 & S = 0A \cup C \cup 1S           &\\
      & 2 & A = 0A \cup 11S                 & I.3 \to I.2\\
      & 4 & C = 0^+D                        & I.4 \to \text{Lema de Arden}\\
      & 5 & D = 10F                         & I.6 \to I.5\\
      & 7 & F = 0F \cup 11F \cup 1 \cup \ve & I.8,I.9 \to I.7\\
  \midrule
  III & 1 & S = 0A \cup C \cup 1S         &\\
      & 2 & A = 0^*11S                    & I.2 \to \text{Lema de Arden}\\
      & 4 & C = 0^+10F                    & II.5 \to II.4\\
      & 7 & F = (0 \cup 11)^*(1 \cup \ve) & II.7 \to \text{Lema de Arden}\\
  \midrule
   IV & 1 & S = (1 \cup 0^+11)S \cup C         & III.2 \to III.1,\ \text{Fatoração}\\
      & 4 & C = 0^+10(0 \cup 11)^*(1 \cup \ve) &\\
  \midrule
    V & 1 & S = (1 \cup 0^+11)^*C              & IV.1 \to \text{Lema de Arden}\\
      & 4 & C = 0^+10(0 \cup 11)^*(1 \cup \ve) &\\
  \midrule
   VI & 1 & S = (1 \cup 0^+11)^*0^+10(0 \cup 11)^*(1 \cup \ve) & V.4 \to V.1\\
  \bottomrule
  \end{array}
  $$
\end{tcolorbox}
%=========================================================================
\end{document}
%
