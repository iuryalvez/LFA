\documentclass[12pt]{article}

% ----------------------------------------- Dados do discente
% Insira os seus dados e do exercício escolhido:
\def\discente{Iury Alexandre Alves Bo}
\def\matricula{202103735}
\def\ua{15}
\def\up{14}
\def\myling{{26}} % Informe o número da linguagem selecionada.

% ------------------------------------------ Babel & Geometry
\usepackage[brazil]{babel}
\usepackage[T1]{fontenc}
\usepackage[utf8]{inputenc}
\usepackage[a4paper,top=1.0cm,bottom=2.0cm,left=1.5cm,right=1.5cm]{geometry}
%
\usepackage{xcolor}
\usepackage{enumitem}
\usepackage{amsmath,amsfonts,amssymb,mathtools}
\usepackage[breakable]{tcolorbox}
%
\newcommand{\ve}{\ensuremath{\varepsilon}}

\begin{document}
% ------------------------------------------------- Cabeçalho
 \begin{tcolorbox}[rounded corners, colback=blue!3, colframe=blue!40!black]
  \footnotesize\textbf{Universidade Federal de Goiás -- UFG}\hfill \textsc{Linguagens Formais e Autômatos -- 2022/2}\\
  \footnotesize\textbf{Instituto de Informática -- INF\hfill Prof. Humberto J. Longo} -- \scriptsize\texttt{longo@inf.ufg.br}
 \end{tcolorbox}\bigskip
%
% ------------------------------------------------- Atividade
\begin{tcolorbox}[rounded corners, colback=blue!2, colframe=blue!40!black, title=\textbf{Atividade AA-\ua}]
 Nesta tarefa deve-se converter a gramática $G_n^5$, obtida na atividade avaliativa anterior (AA-\up\ relativa a transformações em GLC's), para a forma normal de Chomsky \textbf{OU} para a forma normal de Greibach. (Cada aluno(a) deve consultar na descrição da atividade AA--\up, na disciplina INF0333A da plataforma Turing, qual é a linguagem associada ao seu número de matrícula. A descrição da linguagem está disponível no arquivo ``Lista de linguagens livres de contexto'' da Seção ``Coletânea de exercícios''.)
\end{tcolorbox}\bigskip

%
% ------------------------------------ Resolução do exercício
%=========================================================================
\begin{tcolorbox}[rounded corners, colback=yellow!5, colframe=red!40!black, title={\discente\ (\matricula)}]
\begin{itemize}
%- - - - - - - - - - - - - - - - - - - - - - - - - - - - - - - 
  \item  $\mathcal{L}_{\myling} = \{w \in \Sigma^* = {0,1}^* \mid 0^m1^n0^q$, $m = 1 \Longrightarrow n = q$, $m,n \in \mathbf{N}\}$.
%- - - - - - - - - - - - - - - - - - - - - - - - - - - - - - - - 
  \item Gramática $G_{\myling}^2$ (obtida na AA-\up) que gera as cadeias da linguagem $\mathcal{L}_{\myling}$:\\
  $G_{\myling}^2=(V,\Sigma,P,S)=(\{A,B,C,S\},\{0,1\},P,S)$, com
    \[P=\left\{
     \begin{aligned}
      S & \to 0A \mid B \mid 00C \mid \ve, \\
      A & \to 1A0 \mid 10, \\
      B & \to 1B \mid B0 \mid 1 \mid 0, \\
      C & \to 0C \mid 1B \mid B0 \mid 1 \mid 0
     \end{aligned}
    \right\}\]
%- - - - - - - - - - - - - - - - - - - - - - - - - - - - - - - 
 \end{itemize}
\end{tcolorbox}\bigskip

%=========================================================================
\begin{tcolorbox}[breakable,rounded corners, colback=yellow!5, colframe=red!40!black, title={Forma normal de Chomsky.}]
%-----------------------------------------------
\begin{itemize}
  \item Gramática $G_{\myling}^3$ na forma normal de Chomsky, obtida a partir de $G_{\myling}^2$, que gera as cadeias da linguagem $\mathcal{L}_{\myling}$:\\
  $G_{\myling}^3=(V,\Sigma,P,S)=(\{S,A,B,C,D,S_0,S_1,S_{00}\},\{0,1\},P,S)$, com
    \[P=\left\{
     \begin{aligned}
      S & \to S_0A \mid B \mid S_{00}C \mid \ve, \\
      A & \to DS_0 \mid S_1S_0, \\
      B & \to S_1B \mid BS_0 \mid 1 \mid 0, \\
      C & \to S_0C \mid S_1B \mid BS_0 \mid 1 \mid 0, \\
      D & \to S_1A, \\
      S_0 & \to 0,\\
      S_{00} & \to S_0S_0, \\
      S_1 & \to 1
     \end{aligned}
    \right\}\]
\end{itemize}
\end{tcolorbox}
%=========================================================================
\end{document}
%
