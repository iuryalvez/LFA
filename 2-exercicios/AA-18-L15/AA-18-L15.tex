\documentclass[12pt]{article}

% ----------------------------------------- Dados do discente
% Insira os seus dados e do exercício escolhido:
\def\discente{Iury Alexandre Alves Bo}
\def\matricula{202103735}
\def\ua{18}
\def\myling{{15}} % Informe o número da linguagem selecionada.

% ------------------------------------------ Babel & Geometry
\usepackage[brazil]{babel}
\usepackage[T1]{fontenc}
\usepackage[utf8]{inputenc}
\usepackage[a4paper,top=1.0cm,bottom=2.0cm,left=1.5cm,right=1.5cm]{geometry}
%
\usepackage{xcolor}
\usepackage{enumitem}
\usepackage{amsmath,amsfonts,amssymb,mathtools}
\usepackage[breakable]{tcolorbox}
%
% -------------------------------------------------- New commands
%\newcommand{\tvs}{{\large\bf \textvisiblespace}}
\newcommand\vs[1][.8em]{% Visible space!
 \makebox[#1]{%
  \kern.07em
  \vrule height.5ex
  \hrulefill
  \vrule height.5ex
  \kern.07em
 }
}
\newcommand{\ve}{\ensuremath{\varepsilon}}

\begin{document}
% ------------------------------------------------- Cabeçalho
 \begin{tcolorbox}[rounded corners, colback=blue!3, colframe=blue!40!black]
  \footnotesize\textbf{Universidade Federal de Goiás -- UFG}\hfill \textsc{Linguagens Formais e Autômatos -- 2022/2}\\
  \footnotesize\textbf{Instituto de Informática -- INF\hfill Prof. Humberto J. Longo} -- \scriptsize\texttt{longo@inf.ufg.br}
 \end{tcolorbox}\bigskip
%
% ------------------------------------------------- Atividade
\begin{tcolorbox}[rounded corners, colback=blue!2, colframe=blue!40!black, title=\textbf{Atividade AA-\ua}]
 Nesta tarefa deve-se selecionar uma das linguagens listadas na descrição da AA-\ua na disciplina INF0333A da plataforma Turing e demonstrar formalmente (com o auxílio do \emph{Pumping Lemma} para linguagens livres de contexto) que a linguagem escolhida não é livre de contexto. A descrição da linguagem está disponível no arquivo ``Lista de linguagens que não são livres de contexto'' (vide Seção ``Coletânea de exercícios'', na disciplina INF0333A da plataforma Turing).
\end{tcolorbox}\bigskip

%
% ------------------------------------ Resolução do exercício
%=========================================================================
\begin{tcolorbox}[rounded corners, colback=yellow!5, colframe=red!40!black, title={\discente\ (\matricula)}]
 $\mathcal{L}_{\myling} = \{w\in\{0,1\}^*\mid w = 0^n1^m0^{2m},\ m,n \in \mathbb{N}, n > m\}$.
\end{tcolorbox}
%=========================================================================
\begin{tcolorbox}[rounded corners, breakable, colback=yellow!5, colframe=red!40!black, title={A linguagem $\mathcal{L}_{\myling}$ não é livre de contexto.}]
Suponha que $\mathcal{L}_{\myling}$ seja livre de contexto. O \emph{Pumping Lemma} para linguagens livres de contexto garante a existência de $p\in\mathbb{Z}^+$ (\emph{pumping length}), tal que qualquer cadeia $w\in \mathcal{L}_{\myling}$, com $|w|\geqslant p$, pode ser subdividida em subcadeias $u$, $v$, $x$, $y$ e $z$ ($w=uvxyz$) satisfazendo $|vxy|\leqslant p$, $|vy|>0$ ($v\neq\varepsilon$ e/ou $y\neq\varepsilon$) e $uv^ixy^iz\in \mathcal{L}_{\myling}$, para $i\geqslant 0$.\\[10pt]
% - - - - - - - - - - - - - - - - 
Contudo, considere a cadeia $w'=uvxyz=0^{p+1}1^p0^{2p}\in \mathcal{L}_{\myling}$. \\[10pt]
 % - - - - - - - - - - - - - - - -
Se $v$ e $y$ contém, cada um, apenas um símbolo de $\Sigma$ e $x$ conter ao menos um 1, então $w'=uv^2xy^2z\notin \mathcal{L}_{\myling}$, pois $w'$ conterá mais 0's finais do que duas vezes 1's ou 1's maior que a metade dos 0's finais. Se $v$ ou $y$ contém mais de um dos símbolos de $\Sigma$, novamente $w'=uv^2xy^2z\notin \mathcal{L}_{\myling}$, pois $w'$ conterá sequências de 0's e 1's intercalados.\\[10pt]
% - - - - - - - - - - - - - - - - 
 Logo, dadas as contradições ao \emph{Pumping Lemma}, é falsa a suposição de que $\mathcal{L}_{\myling}$ é livre de contexto.\\[10pt]
 - - - - - - - - - - - - - - - - - - - - - - - - - - - - - - - - - - - - - - - - - - - - - - - - - - - - - - - - - - - - - - -\\[10pt]
 OBS: Eu não acredito que posso definir que x contém ao menos um símbolo 1, por isso muito provavelmente a resposta está errada, mas não consegui provar que a linguagem não é livre de contexto sem ser desta forma. \\[10pt]
 Infelizmente não consegui chegar em condições válidas tais que $uvxyz$ contradissesse a linguagem. Por exemplo, se eu assumo $u=\ve$, $x=\ve$, sendo $v$ e $y$ os dois primeiros caracteres de \textbf{qualquer} cadeia e $z$ sendo o resto, as cadeias sempre estarão dentro do estado de aceitação.\\[10pt]
 O que poderia invalidar a cadeia seria $m$ e $2m$, já que $n > m$, encontraríamos uma regra que crescesse $m$ até ser maior que $n$ ou que a quantidade de 0's finais fosse diferente do dobro da quantidade de 1's e vice-versa.\\[10pt] 
 Entretanto $v^i$ e $y^i$ estando no início contribuem para que $n > m$ já que estão na primeira parte da cadeia onde há apenas os 0's iniciais, então se todas as cadeias forem formadas dessa forma, não consigo provar $w$ que não é livre de contexto, apenas para as que eu assumiria que houvesse um $y$ na parte de uma cadeia em que já passou do primeiro 1, por isso $x$ deveria conter ao menos um 1.\\[10pt]
 Posso ter entendido errado mas isso me deixou com muitas dúvidas e não encontrei o que pudesse me ajudar.
\end{tcolorbox}
%=========================================================================
\end{document}
%
