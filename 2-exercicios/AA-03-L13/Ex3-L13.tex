\documentclass[12pt]{article}

% ----------------------------------------- Dados do discente
% Insira os seus dados e do exercício escolhido:
\def\discente{Iury Alexandre Alves Bo}
\def\matricula{202103735}
\def\ua{03}
\def\myling{{13}} % Informe o número da linguagem selecionada.

% ------------------------------------------ Babel & Geometry
\usepackage[brazil]{babel}
\usepackage[T1]{fontenc}
\usepackage[utf8]{inputenc}
\usepackage[a4paper,top=0.5cm,bottom=1.5cm,left=1.5cm,right=1.5cm,nohead,nofoot]{geometry}
%
\usepackage{xcolor}
\usepackage{enumitem}
\usepackage{mathtools}
\usepackage{tcolorbox}

\newcommand{\concatL}{\ensuremath{{\scriptstyle\circ}}}%

\begin{document}
% ------------------------------------------------- Cabeçalho
 \begin{tcolorbox}[rounded corners, colback=blue!3, colframe=blue!40!black]
  \footnotesize\textbf{Universidade Federal de Goiás -- UFG}\hfill \textsc{Linguagens Formais e Autômatos -- 2022/2}\\
  \footnotesize\textbf{Instituto de Informática -- INF\hfill Prof. Humberto J. Longo} -- \scriptsize\texttt{longo@inf.ufg.br}
 \end{tcolorbox}\bigskip
%
% ------------------------------------------------- Atividade
\begin{tcolorbox}[rounded corners, colback=blue!2, colframe=blue!40!black, title=\textbf{Atividade AA-\ua}]
   Nesta tarefa deve-se propôr uma definição por \textbf{Expressões Regulares} para a linguagem selecionada. (Cada aluno deve consultar na descrição da atividade AA-\ua, na disciplina INF0333A da plataforma Turing, qual é a linguagem associada ao seu número de matrícula. A descrição da linguagem está disponível no arquivo ``Lista de linguagens regulares'' da Seção ``Coletânea de exercícios''.)
\end{tcolorbox}\bigskip
%
% ------------------------------------ Resolução do exercício
\begin{tcolorbox}[rounded corners, colback=yellow!5, colframe=red!40!black, title=\textbf{\matricula\ -- \discente}]
 \begin{itemize}[leftmargin=*]
% 
  \item $\mathcal{L}_\myling = \{w\in\Sigma^* = \{0,1\}^* \mid |w| \geq 3$ \textbf{e o terceiro e penúltimo símbolos de $w$ não são 1.}\}.
%
  \item Definição, por expressões regulares, das palavras pertencentes à linguagem $\mathcal{L}_\myling$:
   $$\mathcal{ER}(\mathcal{L}_\myling) =
% * * * * * * * * * * * * * * * * * * * * * * 
    (0\cup1)((00\cup100\cup101)\cup(00\cup10)(0\cup1)^*(00\cup01))
% * * * * * * * * * * * * * * * * * * * * * * 
   .$$
 \end{itemize}
\end{tcolorbox}


% \begin{tcolorbox}[rounded corners, colback=yellow!5, colframe=red!40!black, title=\textbf{\matricula\ -- \discente}]
%  \begin{itemize}[leftmargin=*]
% % 
%   \item $\mathcal{L}_\myling = \{w\in\{0,1\}^*\mid w=0u0, u\in\Sigma^*,$ e $w$ contém exatamente uma ocorrência de $010\}$.
% %
%   \item Definição, por expressões regulares, das palavras pertencentes à linguagem $\mathcal{L}_\myling$:
%    $$\mathcal{ER}(\mathcal{L}_\myling) =
% % * * * * * * * * * * * * * * * * * * * * * * 
%     0(0\cup 11^+0)^*10(0\cup 11^+0)^*
% % * * * * * * * * * * * * * * * * * * * * * * 
%    .$$
%  \end{itemize}
% \end{tcolorbox}

% % MAIS UM EXEMPLO!
% \begin{tcolorbox}[rounded corners, colback=yellow!5, colframe=red!40!black, title=\textbf{\matricula\ -- \discente}]
% \begin{itemize}[leftmargin=*]
% % 
%  \item $\mathcal{L}_\myling = \{w\in\{0,1\}^*\mid w=0u1$ ou $w=1u0$, com $u\in\Sigma^*\}$.
% %
%  \item Definição de $\mathcal{L}_\myling$ por conjuntos regulares:
%   $$\mathcal{ER}(\mathcal{L}_\myling) =
% % * * * * * * * * * * * * * * * * * * * * * * 
%    0(0\cup 1)^*1 \cup 1(0\cup 1)^*0
% % * * * * * * * * * * * * * * * * * * * * * * 
%   .$$
% \end{itemize}
% \end{tcolorbox}
% %
% % -----------------------------------------------
\end{document}
% 
