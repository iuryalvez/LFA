\documentclass[12pt]{article}

% ----------------------------------------- Dados do discente
% Insira os seus dados e do exercício escolhido:
\def\discente{Iury Alexandre Alves Bo}
\def\matricula{202103735}
\def\ua{08}
\def\myling{{36}} % Informe o número da linguagem selecionada.

% ------------------------------------------ Babel & Geometry
\usepackage[brazil]{babel}
\usepackage[T1]{fontenc}
\usepackage[utf8]{inputenc}
\usepackage[a4paper,top=0.5cm,bottom=1.5cm,left=1.5cm,right=1.5cm,nohead,nofoot]{geometry}
%
\usepackage{xcolor}
\usepackage{enumitem}
\usepackage{amsmath,amsfonts,amssymb,mathtools}
\usepackage[breakable]{tcolorbox}
\usepackage{booktabs}

\usepackage{tikz}
\usetikzlibrary{automata,shapes,arrows,positioning}
\tikzset{
 node distance=2.5cm,
 initial text={$M_{\mathcal{L}_\myling}$},
 double distance=1pt,
 every state/.style={semithick,fill=blue!20!white,minimum size=20pt,inner sep=0pt},
 every edge/.style={draw,->,>=stealth,auto,semithick,font=\ttfamily\small},
 accept/.style={accepting,fill=green!20!white},
 reject/.style={fill=red!20!white},
 slopea/.style={sloped,anchor=center,above},
 slopeb/.style={sloped,anchor=center,below}
}

%=-=-=-=-=-=-=-=-=-=-=-=-=-=-=-=-=-= N e w   c o m m a n d s
\newcommand{\ve}{\ensuremath{\varepsilon}}

\begin{document}
% ------------------------------------------------- Cabeçalho
 \begin{tcolorbox}[rounded corners, colback=blue!3, colframe=blue!40!black]
  \footnotesize\textbf{Universidade Federal de Goiás -- UFG}\hfill \textsc{Linguagens Formais e Autômatos -- 2022/2}\\
  \footnotesize\textbf{Instituto de Informática -- INF\hfill Prof. Humberto J. Longo} -- \scriptsize\texttt{longo@inf.ufg.br}
 \end{tcolorbox}\bigskip
%
% ------------------------------------------------- Atividade
\begin{tcolorbox}[rounded corners, colback=blue!2, colframe=blue!40!black, title=\textbf{Atividade AA-\ua}]
  Nesta tarefa deve-se ($i$) propôr um autômato finito determinístico \textbf{mínimo} $D$ que reconheça as cadeias da linguagem selecionada, a partir de $D$ construir uma gramática $G_1$ que gere as cadeias reconhecidas por $D$ e a partir de $G_1$, usando o algoritmo baseado em sistemas de equações, extrair uma expressão regular $\mathcal{R}_1$ que gere as mesmas cadeias geradas por $G_1$; ($ii$) propôr um autômato finito não-determinístico $N$ que reconheça as cadeias da linguagem selecionada e, como no item ($i$), obter a partir de $N$ uma gramática $G_2$ (não necessariamente será regular!) e, a partir desta, obter uma expressão regular $\mathcal{R}_2$. O autômato $N$ pode ser um NFA ou NFA-$\varepsilon$, com pelo menos uma transição não determinística ou uma transição $\varepsilon$. \textbf{Atenção:} NFA's criados a partir do simples acréscimo de transições $\delta(s_i,\varepsilon)=s_i$ ($\varepsilon$-laços) a um DFA não serão considerados corretos, por não permitirem uma avaliação razoável do aprendizado dos conceitos abordados nesta atividade avaliativa. (Cada aluna(o) deve consultar na descrição da atividade AA--\ua, na disciplina INF0333A da plataforma Turing, qual é a linguagem associada ao seu número de matrícula. A descrição da linguagem está disponível no arquivo ``lista de linguagens regulares'' da Seção ``Coletânea de exercícios''.)
\end{tcolorbox}\bigskip

%
% ------------------------------------ Resolução do exercício
\begin{tcolorbox}[rounded corners, colback=yellow!5, colframe=red!40!black, title={\discente\ (\matricula)}]
 \begin{itemize}[leftmargin=*]
% - - - - - - - - - - - - - - - - - - - - - - - - - - - - - - - - - - - -
  \item $\mathcal{L}_\myling = \{w\mid w = |2k|$, $k \in \mathbb{N}$, $w$ não contém $11 \}$.
% - - - - - - - - - - - - - - - - - - - - - - - - - - - - - - - - - - - -
  \item $ER(\mathcal{L}_\myling) = (10\cup 0(10)^*0)^*(\varepsilon \cup 0(10)^*1$.
% - - - - - - - - - - - - - - - - - - - - - - - - - - - - - - - - - - - -
 \end{itemize}
\end{tcolorbox}\bigskip
%=========================================================================
\begin{tcolorbox}[rounded corners, colback=yellow!5, colframe=red!40!black, title={DFA mínimo que reconhece as cadeias de $\mathcal{L}_\myling$}]
 \centering
 \begin{tikzpicture}[
%    transform shape,
%    scale=0.8
 ]
  \node[state,initial, accept]         (s0) {$S$};
  \node[state,below of=s0]     (s1) {$A$};
  \node[state,right of=s0]     (s2) {$B$};
  \node[state,right of=s2, accept]     (s3) {$C$};
%
  \draw (s0) edge[bend left=25] node {0}   (s2)
             edge[bend left=25] node {1}   (s1)
        (s1) edge[bend left=25] node {0}   (s0)
        (s2) edge[bend left=25] node {0}   (s0)
             edge[bend left=25] node {1}   (s3)
        (s3) edge[bend left=25] node {0}   (s2)
        ;
 \end{tikzpicture}
\end{tcolorbox}\bigskip
%=========================================================================
\begin{tcolorbox}[breakable,rounded corners, colback=yellow!5, colframe=red!40!black, title={Gramática $G_1$ que gera as cadeias de $\mathcal{L}_\myling$}]
 \begin{align*}
 G_1 &= (V,\Sigma,P,S)=(\{A,B,C,S\},\{0,1\},P,S),\text{ com:}\\
   P &=
   \left\{\begin{array}{l}
    S \to 0B \mid 1A \mid \varepsilon,\\
    A \to 0S, \\
    B \to 0S \mid 1C,\\
    C \to 0B \mid \varepsilon\\
   \end{array}\right\}.
 \end{align*}
\end{tcolorbox}\bigskip
%=========================================================================
\begin{tcolorbox}[breakable,rounded corners, colback=yellow!5, colframe=red!40!black, title={Extração de expressão regular $\mathcal{R}_1$ da gramática $G$, tal que $\mathcal{L}(\mathcal{R}_1)=\mathcal{L}(G_1)$}]
  $$
  \begin{array}{ccll}
  \toprule
  \textbf{Etapa} & \textbf{Eq.} & \textbf{Expressão} & \textbf{Ação}\\
  \midrule
    I & 1 & S = 0B \cup 1A \cup \ve    &\\
      & 2 & A = 0S                     &\\
      & 3 & B = 0S \cup 1C             &\\
      & 4 & C = 0B \cup \ve            &\\
  \midrule
   II & 1 & S = 0B \cup 10S \cup \ve   & I.2 \to I.1,\texttt{ Fatoração}\\
      & 4 & C = 0(0S \cup 1C) \cup \ve & I.3 \to I.4, \texttt{ Fatoração}\\
  \midrule
  III & 1 & S = 0^*(10S \cup \ve)      & II.1 \to \texttt{ Lema de Arden}\\
      & 4 & C = 00S\cup 01C \cup \ve   & II.4 \to \texttt{ Distributiva}\\
  \midrule
   IV & 1 & S = 0^*10S \cup 0^*        & III.1 \to \texttt{Distributiva}\\
      & 4 & C = (01)^*(00S \cup \ve)   & III.4 \to \texttt{ Lema de Arden}\\
  \midrule
    V & 1 & S = (0^*10)^*0^*           & IV.1 \to IV.1 \texttt{ Lema de Arden} \\
      & 4 & C = (01)^*00S \cup (01)^*  & IV.4 \to IV.4 \texttt{ Distributiva} \\
  \midrule
   VI & 4 & C = (01)^*00(0^*10)^*0^* \cup (01)^*  & IV.4 \to IV.1,  \texttt{ Fatoração} \\
  \bottomrule
  \end{array}
  $$
\end{tcolorbox}\bigskip
%=========================================================================
\begin{tcolorbox}[breakable,rounded corners, colback=yellow!5, colframe=red!40!black, title={NFA que reconhece as cadeias de $\mathcal{L}_\myling$}]
 \centering
 \begin{tikzpicture}
  [
   initial text={$N_\myling$},
   accept/.style={accepting,fill=green!20!white},
   reject/.style={fill=red!20!white},
%    node distance=2.cm,
%    transform shape,
%    scale=1.2
  ]
  \node[state,initial]                 (s0) {$S$};
  \node[state,below of=s0]             (s1) {$A$};
  \node[state,right of=s0, accept]     (s2) {$B$};
  \node[state,right of=s2]             (s3) {$C$};
  \node[state,below of=s3,accept]      (s4) {$D$};
%
  \draw (s0) edge[bend left=25] node {$\varepsilon$}   (s2)
             edge[bend left=25] node {1}   (s1)
        (s1) edge[bend left=25] node {0}   (s0)
        (s2) edge[bend left=25] node {0}   (s3)
             edge[bend left=35] node {1}   (s1)
        (s3) edge[bend left=25] node {0}   (s2)
             edge               node {1}   (s4)
        (s4) edge[bend left=10] node {0}   (s1)
        ;
   \end{tikzpicture}
\end{tcolorbox}\bigskip
%=========================================================================
\begin{tcolorbox}[breakable,rounded corners, colback=yellow!5, colframe=red!40!black, title={Gramática $G_2$ que gera as cadeias da linguagem $\mathcal{L}_{\myling}$}]
\begin{align*}
 G_2 &=(V,\Sigma,P,S)=(\{A,B,C,S\},\{0,1\},P,S), \text{ com}\\
   P &=
   \left\{\begin{array}{l}
    S \to 1A \mid \varepsilon B,\\
    A \to 0S,\\
    B \to 0C \mid 1A \mid \ve, \\
    C \to 0B \mid 1D,\\
    D \to 0A \mid \ve\\
   \end{array}\right\}
   =
   \left\{\begin{array}{l}
    S \to 1A \mid B,\\
    A \to 0S,\\
    B \to 0C \mid 1A \mid \varepsilon, \\
    C \to 0B \mid 1D,\\
    D \to 0A \mid \ve \\
   \end{array}\right\}.
 \end{align*}
\end{tcolorbox}\bigskip
%%=========================================================================
\begin{tcolorbox}[breakable,rounded corners, colback=yellow!5, colframe=red!40!black, title={Extração de expressão regular $\mathcal{R}_2$ da gramática $G_2$, tal que $\mathcal{L}(\mathcal{R}_2)=\mathcal{L}(G_2)$}]
  $$
  \begin{array}{ccll}
  \toprule
  \textbf{Etapa} & \textbf{Eq.} & \textbf{Expressão} & \textbf{Ação}\\
  \midrule
    I & 1 & S = 1A \cup B                &\\
      & 2 & A = 0S                       &\\
      & 3 & B = 0C \cup 1A \cup \ve      &\\
      & 4 & C = 0B \cup 1D               &\\
      & 5 & D = 0A \cup \ve              &\\
  \midrule
   II & 1 & S = 10S \cup B               & I.2 \to I.1, \texttt{ Fatoração} \\
      & 3 & B = 0C \cup 10S \cup \ve     & I.2 \to I.3, \texttt{ Fatoração} \\
      & 4 & C = 0B \cup 100S \cup 1       & I.2 \to I.5, \texttt{ Fatoração e } I.5 \to I.4, \texttt{ Fatoração}\\
  \midrule
  III & 1 & S = (10)^*B                  & II.1 \to \texttt{Lema de Arden}\\
      & 3 & B = 00B \cup 0100S \cup 01   & II.4 \to II.3,\texttt{ 
 Fatoração}\\
  \midrule
   IV & 1 & S = (10)^*B                  & \\
      & 3 & B = (00)^*0100S \cup (00)^*01& III.3 \to \texttt{Lema de Arden}\\
  \midrule
    V & 3 & B = (00)^*0100(10)^*B \cup (00)^*01 & IV.3 \to IV.1,\texttt{ Fatoração}\\
  \midrule
   VI & 3 & B = ((00)^*0100(10)^*)^*(00)^*01 & V.3 \to \texttt{Lema de Arden}\\
  \bottomrule
  \end{array}
  $$
\end{tcolorbox}
%=========================================================================
\end{document}
%
