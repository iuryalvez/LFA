\documentclass[12pt]{article}

% ----------------------------------------- Dados do discente
% Insira os seus dados e do exercício escolhido:
\def\discente{Iury Alexandre Alves Bo}
\def\matricula{202103735}
\def\ua{09}
\def\myling{{21}} % Informe o número da linguagem selecionada.

% ------------------------------------------ Babel & Geometry
\usepackage[brazil]{babel}
\usepackage[T1]{fontenc}
\usepackage[utf8]{inputenc}
\usepackage[a4paper,top=0.5cm,bottom=1.5cm,left=1.5cm,right=1.5cm,nohead,nofoot]{geometry}
%
\usepackage{xcolor}
\usepackage{enumitem}
\usepackage{amsmath,amsfonts,amssymb,mathtools}
\usepackage{tcolorbox}
\usepackage{booktabs}

\usepackage{tikz}
\usetikzlibrary{automata,shapes,arrows,positioning}
\tikzset{
 node distance=2.5cm,
 initial text={$M_{\mathcal{L}_\myling}$},
 double distance=1pt,
 every state/.style={semithick,fill=blue!20!white,minimum size=20pt,inner sep=0pt},
 every edge/.style={draw,->,>=stealth,auto,semithick,font=\ttfamily\small},
 accept/.style={accepting,fill=green!20!white},
 reject/.style={fill=red!20!white},
 slopea/.style={sloped,anchor=center,above},
 slopeb/.style={sloped,anchor=center,below}
}

%=-=-=-=-=-=-=-=-=-=-=-=-=-=-=-=-=-= N e w   c o m m a n d s
\newcommand{\ve}{\ensuremath{\varepsilon}}

\begin{document}
% ------------------------------------------------- Cabeçalho
 \begin{tcolorbox}[rounded corners, colback=blue!3, colframe=blue!40!black]
  \footnotesize\textbf{Universidade Federal de Goiás -- UFG}\hfill \textsc{Linguagens Formais e Autômatos -- 2022/2}\\
  \footnotesize\textbf{Instituto de Informática -- INF\hfill Prof. Humberto J. Longo} -- \scriptsize\texttt{longo@inf.ufg.br}
 \end{tcolorbox}\bigskip
%
% ------------------------------------------------- Atividade
\begin{tcolorbox}[rounded corners, colback=blue!2, colframe=blue!40!black, title=\textbf{Atividade AA-\ua}]
  Nesta tarefa deve-se propôr um autômato finito não determinístico $N$ que reconheça as cadeias da linguagem selecionada, a partir de $N$ deve-se usar o algoritmo baseado em GNFA's para extrair uma expressão regular que gere as cadeias da linguagem reconhecida pelo NFA. O autômato $N$ pode ser um NFA ou NFA-$\varepsilon$, com pelo menos uma transição não determinística ou transição $\varepsilon$. \textbf{Atenção:} NFA's criados a partir do simples acréscimo de transições $\delta(s_i,\varepsilon)=s_i$ ($\varepsilon$-laços) a um DFA não serão considerados corretos, por não permitirem uma avaliação razoável do aprendizado dos conceitos abordados nesta atividade avaliativa. (Cada aluna(o) deve consultar na descrição da atividade AA--\ua, na disciplina INF0333A da plataforma Turing, qual é a linguagem associada ao seu número de matrícula. A descrição da linguagem está disponível no arquivo ``lista de linguagens regulares'' da Seção ``Coletânea de exercícios''.)
\end{tcolorbox}\bigskip

%
% ------------------------------------ Resolução do exercício
\begin{tcolorbox}[rounded corners, colback=yellow!5, colframe=red!40!black, title={\discente\ (\matricula)}]
 \begin{itemize}[leftmargin=*]
% - - - - - - - - - - - - - - - - - - - - - - - - - - - - - - - - - - - -
  \item $\mathcal{L}_\myling = \{w \in \Sigma = \{0,1\}^* \mid |w|_{00} \geqslant 1$ e $|w|_{11} = 0$\}.
% - - - - - - - - - - - - - - - - - - - - - - - - - - - - - - - - - - - -
  \item $ER(\mathcal{L}_\myling) = (0\cup 10)(10)^*0(0\cup 10)^*(\ve\cup 1)$.
% - - - - - - - - - - - - - - - - - - - - - - - - - - - - - - - - - - - -
 \end{itemize}
\end{tcolorbox}\bigskip
%=========================================================================
\begin{tcolorbox}[rounded corners, colback=yellow!5, colframe=red!40!black, title={Autômato finito não determinístico que reconhece as cadeias de $\mathcal{L}_\myling$}]
 \centering
 \begin{tikzpicture}[
  initial text={$N_\myling$},
%    transform shape,
%    scale=0.8
 ]
    \node[state,initial]                      (s0) {$s_0$};
    \node[state,below of=s0]                  (s1) {$s_1$};
    \node[state,right of=s0]                  (s2) {$s_2$};
    \node[state,right of=s1]                  (s3) {$s_3$};
    \node[state,right of=s2]                  (s4) {$s_4$};
    \node[state,right of=s3]                  (s5) {$s_5$};
    \node[state,right of=s4, accept]          (s6) {$s_6$};
  %
    \draw (s0) edge                         node {1} (s1)
               edge                         node {0} (s2)
          (s1) edge                         node {0} (s2)
          (s2) edge [bend left=20]          node {1} (s3)
               edge                         node {0} (s4)
          (s3) edge [bend left=20]          node {0} (s2)
          (s4) edge [loop above]            node {0} (s4)
               edge [bend left=20]          node {1} (s5)
               edge                         node {0,\ve} (s6)
          (s5) edge [bend left=20]          node {0} (s4)
          ;
 \end{tikzpicture}
\end{tcolorbox}\bigskip
%=========================================================================
\begin{tcolorbox}[rounded corners, colback=yellow!5, colframe=red!40!black, title={GNFA $G_0$ obtido a partir do NFA $N_\myling$}]
\centering
\begin{tikzpicture}[
%    transform shape,
%    scale=0.8
 ]
    \node[state,initial]                      (si) {$I$};
    \node[state,right of=si]                  (s0) {$s_0$};
    \node[state,below of=s0]                  (s1) {$s_1$};
    \node[state,right of=s0]                  (s2) {$s_2$};
    \node[state,right of=s1]                  (s3) {$s_3$};
    \node[state,right of=s2]                  (s4) {$s_4$};
    \node[state,right of=s3]                  (s5) {$s_5$};
    \node[state,right of=s4]                  (s6) {$s_6$};
    \node[state,right of=s6,accept]           (sf) {$F$};
  %
    \draw 
          (si) edge                         node {\ve} (s0)  
          (s0) edge                         node {1} (s1)
               edge                         node {0} (s2)
          (s1) edge                         node {0} (s2)
          (s2) edge [bend left=20]          node {1} (s3)
               edge                         node {0} (s4)
          (s3) edge [bend left=20]          node {0} (s2)
          (s4) edge [loop above]            node {0} (s4)
               edge [bend left=20]          node {1} (s5)
               edge                         node {0} (s6)
          (s5) edge [bend left=20]          node {0} (s4)
          (s6) edge                         node {\ve} (sf)
          ;
 \end{tikzpicture}
\end{tcolorbox}\bigskip
%=========================================================================
\begin{tcolorbox}[rounded corners, colback=yellow!5, colframe=red!40!black, title={GNFA $G_2$ obtido a partir do GNFA $G_0$ após a remoção dos estados $s_0$ e $s_6$}]
\centering
\begin{tikzpicture}[
%    transform shape,
%    scale=0.8
 ]
    \node[state,initial]                      (si) {$I$};
    \node[state,below right of=si]            (s1) {$s_1$};
    \node[state,above right of=si]            (s2) {$s_2$};
    \node[state,below right of=s2]                  (s3) {$s_3$};
    \node[state,right of=s2]                  (s4) {$s_4$};
    \node[state,below right of=s4]            (s5) {$s_5$};
    \node[state,right of=s4,accept]           (sf) {$F$};
  %
    \draw 
          (si) edge [bend left=30]          node {0} (s2)
               edge [bend right=30]         node {1} (s1)
          (s1) edge                         node {0} (s2)
          (s2) edge [bend left=20]          node {1} (s3)
               edge                         node {0} (s4)
          (s3) edge [bend left=20]          node {0} (s2)
          (s4) edge [loop above]            node {0} (s4)
               edge [bend left=20]          node {1} (s5)
               edge                         node {0} (sf)
          (s5) edge [bend left=20]          node {0} (s4)
          ;
 \end{tikzpicture} 
\end{tcolorbox}\bigskip
%=========================================================================
\begin{tcolorbox}[rounded corners, colback=yellow!5, colframe=red!40!black, title={GNFA $G_5$ obtido a partir do GNFA $G_2$ após a remoção dos estados $s_1$, $s_3$ e $s_5$}]
\centering
\begin{tikzpicture}[
%    transform shape,
%    scale=0.8
 ]
    \node[state,initial]                      (si) {$I$};
    \node[state,right of=si]                  (s2) {$s_2$};
    \node[state,right of=s2]                  (s4) {$s_4$};
    \node[state,right of=s4,accept]           (sf) {$F$};
  %
    \draw 
          (si) edge                         node {$0 \cup 10$} (s2)
          (s2) edge [loop above]            node {10} (s2)
               edge                         node {0} (s4)
          (s4) edge [loop above]            node {$0 \cup 10$} (s4)
               edge                         node {0} (sf)
          ;
 \end{tikzpicture}
\end{tcolorbox}\bigskip
%=========================================================================
\begin{tcolorbox}[rounded corners, colback=yellow!5, colframe=red!40!black, title={GNFA $G_7$ obtido a partir do GNFA $G_5$ após a remoção dos estados $s_2$ e $s_4$}]
\centering
\begin{tikzpicture}[
%    transform shape,
%    scale=0.8
 ]
    \node[state,initial]                      (si) {$I$};
    \node[state,right of=si,xshift=5cm,accept](sf) {$F$};
  %
    \draw 
          (si) edge node {$0\cup10(10)^*0(0\cup10)^*0$} (sf)
          ;
 \end{tikzpicture}
\end{tcolorbox}\bigskip
%=========================================================================
\end{document}
%
