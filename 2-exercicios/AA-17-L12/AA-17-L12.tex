\documentclass[12pt]{article}

% ----------------------------------------- Dados do discente
% Insira os seus dados e do exercício escolhido:
\def\discente{Iury Alexandre Alves Bo}
\def\matricula{202103735}
\def\ua{17}
\def\up{16}
\def\myling{{12}} % Informe o número da linguagem selecionada.

% ------------------------------------------ Babel & Geometry
\usepackage[brazil]{babel}
\usepackage[T1]{fontenc}
\usepackage[utf8]{inputenc}
\usepackage[a4paper,top=1.0cm,bottom=2.0cm,left=1.5cm,right=1.5cm]{geometry}
%
\usepackage{marvosym} % \Smiley
\usepackage{xcolor}
\usepackage{enumitem}
\usepackage{amsmath,amsfonts,amssymb,mathtools}
\usepackage[breakable]{tcolorbox}
%
\usepackage{tikz}
\usetikzlibrary{automata,arrows,positioning,shapes}
\tikzset{
 node distance=3cm,
 initial text={$P_{\myling}$},
 double distance=1pt,
 every state/.style={semithick,fill=blue!20!white,minimum size=20pt,inner sep=0pt},
 every edge/.style={draw,->,>=stealth,auto,semithick,font=\ttfamily\small}
}
%
% -------------------------------------------------- New commands
%\newcommand{\tvs}{{\large\bf \textvisiblespace}}
\newcommand\vs[1][.8em]{% Visible space!
 \makebox[#1]{%
  \kern.07em
  \vrule height.5ex
  \hrulefill
  \vrule height.5ex
  \kern.07em
 }
}
\newcommand{\ve}{\ensuremath{\varepsilon}}

\begin{document}
% ------------------------------------------------- Cabeçalho
 \begin{tcolorbox}[rounded corners, colback=blue!3, colframe=blue!40!black]
  \footnotesize\textbf{Universidade Federal de Goiás -- UFG}\hfill \textsc{Linguagens Formais e Autômatos -- 2022/2}\\
  \footnotesize\textbf{Instituto de Informática -- INF\hfill Prof. Humberto J. Longo} -- \scriptsize\texttt{longo@inf.ufg.br}
 \end{tcolorbox}\bigskip
%
% ------------------------------------------------- Atividade
\begin{tcolorbox}[rounded corners, colback=blue!2, colframe=blue!40!black, title=\textbf{Atividade AA-\ua}]
 Nesta tarefa deve ser escolhida \textbf{uma} das seguintes opções:
 \begin{enumerate}
  \item Propor uma gramática livre de contexto $G_n$ que gere as cadeias da linguagem selecionada e construir, segundo um dos algoritmos apresentados nas aulas, um PDA que reconheça as cadeias da linguagem gerada pela gramática.
  \item A partir do PDA $P_n$ obtido na atividade ``AA-16 : Autômatos com pilha'', obter uma gramática que gere a linguagem aceita pelo PDA.
 \end{enumerate}
 (Cada aluna(o) deve consultar na descrição da atividade AA--\up, na disciplina INF0333A da plataforma Turing, qual é a linguagem associada ao seu número de matrícula. A descrição da linguagem está disponível no arquivo ``Lista de linguagens livres de contexto'' da Seção ``Coletânea de exercícios''.)
\end{tcolorbox}\bigskip

%
% ------------------------------------ Resolução do exercício
%================================================================
\begin{tcolorbox}[rounded corners, colback=yellow!5, colframe=red!40!black, title={OPÇÂO 2: Gramática $G^2_{\myling}$ obtida a partir do PDA $P_{\myling}$}]
 \begin{itemize}
 \item $\mathcal{L}_{\myling} = \{w \in \Sigma^* = \{0,1\}^* \mid |w|_0 = 2\cdot|w|_1\}$.
 \item PDA $P_{\myling}$ que reconhece as cadeias da linguagem $\mathcal{L}_{\myling}$:
   \begin{center}
\hspace*{-1.5cm}
\begin{tikzpicture}[
  accept/.style={accepting,fill=green!20!white},
  reject/.style={accepting,fill=red!20!white},
  slopea/.style={sloped,anchor=center,above},
  slopeb/.style={sloped,anchor=center,below},
  transform shape,
  scale=0.9
 ]
  \node[initial,state]            (s0) {$s_0$};
  \node[state,right of=s0]        (s1) {$s_1$};
  \node[state,right of=s1]        (s2) {$s_2$};
  \node[state, accept, below of=s1](s3) {$s_3$};

  \path (s0) edge node {$\ve,\ve\to \$$}  (s1)
        (s1) edge [bend left=15,above] node [align=left]{$1,\ve\to 0$ \\ 
             [-3pt] $0,\ve\to1$} (s2) 
             edge [loop above] node [align=left]{$1,1\to \ve$ \\ [-3pt]
                                     $0,0\to \ve$ } (s1)
             edge node {$\vs,\$ \to \ve$} (s3) 
        (s2) edge [bend left=15, below] node {$\ve,\ve\to 0$}(s1)
             edge [loop right] node [align=left] {$1,1\to \ve$ \\ [-3pt] $0,0\to \ve$} (s2) 
        ;
\end{tikzpicture}
\end{center}
\item Gramática $G^2_{\myling}$ obtida a partir do PDA $P_{\myling}$ que gera as cadeias da linguagem $\mathcal{L}_{\myling}$: 
\item $G^2_{\myling}=(V,\Sigma,P,S)=(\{S,\left[S_1,1,S_1\right],
      \left[S_1,0,S_1\right],\left[S_2,1,S_2\right],\left[S_2,0,S_2\right], \left[S_1,\$,S_2\right],\left[S_0,\ve,S_1\right],\left[S_1,\ve,S_2\right], \left[S_1,\ve,S_2\right], \left[S_2,\ve,S_1\right]\},\{0,1\},P,S)$, com
\begin{center}
    $
     P =
     \left\{\begin{array}{r@{\;\to\;}l}
      S   &  \left[S_0,1,q\right],\\\
      \left[S_1,1,S_1\right] & 1, \\
      \left[S_1,0,S_1\right] & 0, \\
      \left[S_2,1,S_2\right] & 1, \\
      \left[S_2,0,S_2\right] & 0, \\
      \left[S_1,\$,S_2\right] & \vs, \\
      \left[S_0,\ve,S_1\right] & \ve\left[S_1,\$,q\right], \\
      \left[S_1,\ve,S_2\right] & 1\left[S_2,0,q\right], \\
      \left[S_1,\ve,S_2\right] & 0\left[S_2,1,q\right], \\
      \left[S_2,\ve,S_1\right] & \ve\left[S_1,0,q\right] 
     \end{array}\right\}.
    $
\end{center}
\end{itemize}
\end{tcolorbox}
%=========================================================================
\end{document}
%
